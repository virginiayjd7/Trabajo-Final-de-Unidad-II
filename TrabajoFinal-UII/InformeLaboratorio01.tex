%
\documentclass[%
 reprint,
 amsmath,amssymb,
 aps,
]{revtex4-1}

\usepackage{graphicx}% Include figure files
\usepackage{dcolumn}% Align table columns on decimal point
\usepackage{bm}% bold math


\begin{document}



\title{Comparación BD NoSQL}
\author{Virginia Aquino Huallpa}
\author{Arlyn Cotrado Coaquira}
\author{Sharon Sosa Bedoya}
\author{Marlon Villegas Arando}
\affiliation{%
 Universidad Privada de Tacna \textbackslash Facultad de Ingenieria \textbackslash Escuela Profesional de Ingenieria de Sistemas
}%


\begin{abstract}
\begin{center}
\textbf{Resumen}
\end{center}

En el presente articulo se relata la comparación hecha de bases de datos NoSQL, describiéndolas y analizando su importancia, así como las definiciones de tipos de bases de datos NoSQL, con el fin de proporcionar un punto de partida para los trabajos en esta área. Y también la creación de una base de datos, inserción y consultas de datos NoSQL mediante Docker.\\

\textbf{Palabras clave:}   NoSQL, Bases de datos, Docker.\\

\begin{center}
\textbf{Abstract}
\end{center}
In the present article the comparison made of NoSQL databases is described, describing them and analyzing their importance, as well as the definitions of NoSQL database types, in order to provide a starting point for work in this area. And also the creation of a database, insertion and queries of NoSQL data through Docker.\\
\textbf{Keywords:}  NoSQL, Databases, Docker.\\

\end{abstract}



\maketitle

%\tableofcontents

\section {Introducción}\label{sec:1}

Dia a día el manejo de la información se hace más complejo; diferentes factores hacen que las personas involucradas en el área busquen tecnologías que le ayuden con este problema. Las bases de datos relacionales son las mas comunes, pero en los últimos años ha aumentado el interés por las bases de datos NoSQL (Not only SQL), un nuevo conjunto de tecnologías que pueden contribuir al manejo de información.
Por lo anterior, el presente documento hace una revisión de las tecnologías NoSQL, haciendo posible hacer una comparación.\\

\par El resto de este articulo está organizado de la siguiente manera. En la Sección 2 se muestra los matriales y métodos usados para el desarrollo de este articulo. La Sección 3 se explican los resultados. Y finalmente, las conclusiones están en la Sección 4.



%-----------------------------------------------------------------
\section{Materiales y Métodos}\label{sec:2}


%-----------------------------------------------------------------
\section {Resultados}\label{sec:3}
\subsection{Creacion de base de datos NoSQL}
\subsection{Inserción y consulta de datos}
\subsection{Comparación}
\subsection{Documental}
\subsection{Clave-Valor}
\subsection{Grafos}
\subsection{Tabular (Column-Store)}

\subsection{Comparación}
%-----------------------------------------------------------------
\section{Discusión y Conclusiones}\label{sec:4}




% Bibliografia.
%-----------------------------------------------------------------

\bibliographystyle{plain}
\bibliography{Bibliografia}

\end{document}
