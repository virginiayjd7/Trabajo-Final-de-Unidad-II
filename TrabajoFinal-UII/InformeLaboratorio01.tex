%
\documentclass[%
 reprint,
 amsmath,amssymb,
 aps,
]{revtex4-1}

\usepackage{graphicx}% Include figure files
\usepackage{dcolumn}% Align table columns on decimal point
\usepackage{bm}% bold math


\begin{document}



\title{Comparación BD NoSQL}
\author{Virginia Aquino Huallpa}
\author{Arlyn Cotrado Coaquira}
\author{Sharon Sosa Bedoya}
\author{Marlon Villegas Arando}
\affiliation{%
 Universidad Privada de Tacna \textbackslash Facultad de Ingenieria \textbackslash Escuela Profesional de Ingenieria de Sistemas
}%


\begin{abstract}
\begin{center}
\textbf{Resumen}
\end{center}

En este articulo aprenderemos sobre el concepto de SQL y NoSQL, como también sus ventajas y desventajas, lo cual nos permitirá aclarar las diferencias que existen entre ambas y elegir la mejor base de datos que se adapte a su negocio.\\

\textbf{Palabras clave:}   sql, nosql, base de datos.\\

\begin{center}
\textbf{Abstract}
\end{center}
In this article we will learn about the concept of SQL and NoSQL, as well as its advantages and disadvantages, which will allow us to clarify the differences between the two and choose the best database that suits your business.\\
\textbf{Keywords:}   sql, nosql, databases.\\

\end{abstract}



\maketitle

%\tableofcontents

\section {Introducción}\label{sec:1}



%-----------------------------------------------------------------
\section{Materiales y Métodos}\label{sec:2}


%-----------------------------------------------------------------
\section {Resultados}\label{sec:3}

%-----------------------------------------------------------------
\section {Discusión}\label{sec:4}


%-----------------------------------------------------------------
\section{Conclusiones}\label{sec:5}



% Bibliografia.
%-----------------------------------------------------------------

\bibliographystyle{plain}
\bibliography{Bibliografia}

\end{document}
